% report.tex

\documentclass[a4paper,11pt]{article}
\usepackage{hyperref}
\usepackage{enumitem}
\usepackage{varwidth}
\usepackage{tasks}
% Import packages
\usepackage[a4paper]{geometry}
\usepackage[utf8]{inputenc}
\usepackage{amsmath}
\usepackage{amssymb}
\usepackage{enumerate}
\usepackage{geometry}
 \geometry{
 a4paper,
 total={170mm,257mm},
 left=20mm,
 top=20mm,
 }

\usepackage{graphicx}

\usepackage{listings}

% Change enumerate environments you use letters
\renewcommand{\theenumi}{\alph{enumi}}

% Set title, author name and date
\title{Consensus}
\author{Johannes Jørgensen (jgjo),\\ Kevin Skovgaard Gravesen (kegr),\\ Joakim Andreasen (joaan)} 
\date{\today}

\begin{document} 

\maketitle

\subsection*{Introduction}
Consensus is a proof of concept of the Ricart–Agrawala Algorithm. The program is writen in Golang and uses gRPC to pass messages between the clients.
The program is ran by first creating a .env file, with a set of ports to be used, and then running the main.go file multiple times in different terminals.
Each terminal is running a peer which will begin to ask the other peers for permission to access the critical section.

Various improments can be made, including the ability to run the peers on different ip's and to keep the gRPC connections open for longer periods.
It was decided to keep the proof of concept simple, hence the dicision to postpone these improments.

\subsection*{Peers and the critical section}


\subsection*{Only one peer is accessing the critical section at any given time}


\subsection*{Every peer gets access to the critical section}



\newpage
\subsection*{Link to Github repository}

\href{https://github.com/ITU-DISYS2024-CENTRALIZEDSYSTEMS/Consensus}{https://github.com/ITU-DISYS2024-CENTRALIZEDSYSTEMS/Consensus}

\subsection*{Appendix}

\subsubsection*{Logs}
\begin{lstlisting}[basicstyle=\ttfamily\footnotesize]

\end{lstlisting}
\end{document}
